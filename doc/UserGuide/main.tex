\documentclass{report}

%\usepackage{fancyhdr}
%\pagestyle{fancy}

\usepackage{algorithm2e}
\usepackage{amsmath}
\usepackage{amsthm}
\usepackage{balance}
\usepackage{color}
\usepackage{enumitem}
\usepackage{graphicx}
\usepackage{latexsym}
\usepackage{listings}
\usepackage{mathabx}
\usepackage{verbatim}
\usepackage{proof}
\usepackage{subfigure}
%\usepackage[countmax]{subfloat}
%\usepackage{cns-proof}
\usepackage{multirow}
\usepackage{enumitem}
\usepackage{xspace}

\usepackage{float}



\textwidth = 6.8 in
\textheight = 9.3 in
\oddsidemargin = -0.25 in
\evensidemargin = -0.1 in
\leftmargin= -0.3 in
\topmargin = -0.8 in


\usepackage{ifxetex}
\ifxetex
    % XeLaTeX
    \usepackage{polyglossia}
    \usepackage{fontspec}
    \usepackage[]{unicode-math}
\else
    % default: pdfLaTeX
    \usepackage[english]{babel}
    \usepackage[T1]{fontenc}
    \usepackage[utf8]{inputenc}
    \usepackage[babel=true,activate={true,nocompatibility},final,tracking=true,kerning=true,spacing=true]{microtype}
\fi

%%
%% hyperref options
%%
\newif\ifhidelinks
\hidelinksfalse
\ifhidelinks
  \usepackage[hidelinks,bookmarks]{hyperref}
\else
  \usepackage[bookmarks]{hyperref}
\fi

%% Listings-related stuff
\usepackage[scaled]{beramono}
\newcommand\Small{\fontsize{8}{8.2}\selectfont}
\newcommand*\LSTfont{\Small\ttfamily\SetTracking{encoding=*}{-60}\lsstyle}

\lstset{language=C}
\lstset{
  morekeywords={take,emit,repeat,return,map,if,then,else,let,in,letref,letfun,letfunc,
                times,for,neg,
                arr,bit,int}
}
\lstset{
  escapeinside={@}{@},
  boxpos=c,
  basicstyle=\LSTfont,
  captionpos=b,
  numbers=left,
  numbersep=10pt,
  literate={<-}{{$\leftarrow$}}2 
           {>>>}{{$\parc$}}3
           {tt}{{$()$}}2
}

\newenvironment{pseudocode}[1][htb]
  {\renewcommand{\algorithmcfname}{Listing}%
   \begin{algorithm}[#1]%
  }{\end{algorithm}}  

\newcommand{\edmargincomment}[1]{\-\marginpar{\raggedright\footnotesize {\color{red}#1}}}


\setlength{\parskip}{0.12\baselineskip plus 0.1\baselineskip minus 0.1\baselineskip}
\setlength{\parsep}{\parskip}
\setlength{\topsep}{0cm}
\setlength{\parindent}{0cm}

\renewcommand{\textfraction}{0.1}
\renewcommand{\topfraction}{0.95}
\renewcommand{\dbltopfraction}{0.95}
\renewcommand{\floatpagefraction}{0.9}
\renewcommand{\dblfloatpagefraction}{0.9}

\setlength{\floatsep}{16pt plus 4pt minus 4pt}
%% \setlength{\textfloatsep}{14pt plus 4pt minus 4pt}



\newcommand{\edcomment}[2]{{\color{red}\textbf{#1:} #2}}
%% \newcommand{\edcomment}[2]{}
\newcommand{\BR}[1]{\edcomment{BR}{#1}}
\newcommand{\DV}[1]{\edcomment{DV}{#1}}
\newcommand{\GM}[1]{\edcomment{GM}{#1}}
\newcommand{\GS}[1]{\edcomment{GS}{#1}}
\newcommand{\MG}[1]{\edcomment{MG}{#1}}

%% Theorem and definition types
\newtheorem{theorem}{Theorem}
\newtheorem{lemma}{Lemma}
\newtheorem{definition}{Definition}

%% List-related stuff
\setlist{nolistsep}

%% =============== Macros ====================
\def\LANG {{$\mathsf{Blink}$}\xspace}
\def\CORE {{$\lambda_{\mathsf{Blink}}$}\xspace}
\def\LANGIL {{$\mathsf{Blink}_{\mathsf{IL}}$}\xspace}
\def\WPL {{PL$\circ$W}\xspace}

\newcommand{\keyw}[1]{\ensuremath{\textsf{#1}}}

\def\take {{\lstinline|take| }}
\def\emit {{\lstinline|emit| }}

\newcommand{\parc}{\keyw{>\!\!>\!\!>}}
\newcommand{\pipelineparc}{\keyw{|>\!\!>\!\!>|}}
\newcommand{\ST}[3]{\keyw{ST}\ #1\ #2\ #3}
\newcommand{\STC}[3]{\keyw{ST}\ (\keyw{C}\ #1)\ #2\ #3}




\newcommand{\STT}[2]{\keyw{ST}\ \keyw{T}\ #1\ #2}
\newcommand{\bindc}{\keyw{>\!\!>\!=}}

\newcommand{\step}[3]{#1 \overset{#2}{\longrightarrow} #3}
\newcommand{\stepstar}[3]{#1 \overset{#2}{\longrightarrow}^{*} #3}
\newcommand{\twostep}[3]{\begin{array}{rc}& #1 \\ 
  \overset{#2}{\longrightarrow} & #3\end{array}}
\newcommand{\bigstep}[2]{#1\ \Downarrow\ #2}
\newcommand{\bigsteptwoline}[2]{\begin{array}{c} #1\ \ \Downarrow \\ #2\end{array}}
\newcommand{\procstep}[3]{#1\ \ \Downarrow\!\![#2]\ \ #3}
\newcommand{\twoconfig}[2]{\langle #2,\ #1 \rangle}
%% #1 = command
%% #2 = state
\newcommand{\threeconfig}[3]{\langle #3,\ #1,\ #2 \rangle}
%% #1 = command
%% #2 = result/result kind
%% #3 = state
\newcommand{\M}[5]{(#4\ |\ #2\ |\ #3\ |\ #5\ |\ #1)}
%% #1 = result/result kind
%% #2 = input stream
%% #3 = output stream
%% #4 = state
%% #5 = computation
\newcommand{\PM}[8]{(#1\ |\ #2\ |\ #3\ |\ #4\ |\ #5\ |\ #6\ |\ #7\ |\ #8)}
%% #1 = state
%% #2 = input stream
%% #3 = thread buffer
%% #4 = output stream
%% #5 = computation1
%% #6 = computation2
%% #7 = result/result kind1
%% #8 = result/result kind2
\newcommand{\interpTy}[1]{\mathcal{T}(#1)}
\newcommand{\interpExpr}[1]{\mathcal{E}(#1)}
\newcommand{\myvector}[1]{\overrightarrow{#1}}
\newcommand{\Sinit}[1]{S_{\mathsf{init}}(#1)}
\newcommand{\Sdecl}[1]{S_{\mathsf{decl}}(#1)}
\newcommand\Sctx[2]{S#1[#2]}


\newcommand\rimm[1]{{\tt imm}(#1)\xspace}
\newcommand\rcons{{\tt cons}\xspace}
\newcommand{\rskip}{{\tt skip}\xspace}
\newcommand{\ryield}[1]{{\tt yield}(#1)\xspace}
\newcommand{\rdone}[1]{{\tt done}(#1)\xspace}

\newcommand{\ebrack}[1]{{\{\!\mid}#1{\mid\!\}}}

%% \newcommand{\evalinit}{\vdash^{\hspace{-5pt}{\footnotesize{\tt init}}}}
%% \newcommand{\evaltick}{\vdash^{\hspace{-5pt}{\footnotesize{\tt tick}}}}
%% \newcommand{\evalproc}{\vdash^{\hspace{-5pt}{\footnotesize{\tt proc}}}}
\newcommand{\evalinit}{} %% \vdash^{\hspace{-5pt}{\footnotesize{\tt init}}}}
\newcommand{\evaltick}{} %% \vdash^{\hspace{-5pt}{\footnotesize{\tt tick}}}}
\newcommand{\evalproc}{} %% \vdash^{\hspace{-5pt}{\footnotesize{\tt proc}}}}
%% \newcommand{\evalinit}{\vdash^{\hspace{-5pt}{\tt init}}}
%% \newcommand{\evaltick}{\vdash^{\hspace{-5pt}{\tt tick}}}
%% \newcommand{\evalproc}{\vdash^{\hspace{-5pt}{\tt proc}}}
\newcommand{\In}{I}
\newcommand{\Out}{O}

\newcommand{\ol}[1]{\overline{#1}}

\newcommand{\tunit}{\keyw{unit}}
\newcommand{\vunit}{{\tt{()}}}

\newcommand{\stepsto}{\hookrightarrow}

\newcommand\dv[1]{\begin{color}{red}#1\end{color}}
%% ============= End Macros ==================

\begin{document}

\special{papersize=8.5in,11in}
\setlength{\pdfpageheight}{\paperheight}
\setlength{\pdfpagewidth}{\paperwidth}

%% \conferenceinfo{CONF 'yy}{Month d--d, 20yy, City, ST, Country} 
%% \copyrightyear{20yy} 
%% \copyrightdata{978-1-nnnn-nnnn-n/yy/mm} 
%% \doi{nnnnnnn.nnnnnnn}

% Uncomment one of the following two, if you are not going for the 
% traditional copyright transfer agreement.

%\exclusivelicense                % ACM gets excblusive license to publish, 
                                  % you retain copyright

%\permissiontopublish             % ACM gets nonexclusive license to publish
                                  % (paid open-access papers, 
                                  % short abstracts)

%% \titlebanner{banner above paper title}        % These are ignored unless
%% \preprintfooter{short description of paper}   % 'preprint' option specified.

\title{\LANG: User Guide}
%% \subtitle{Subtitle Text, if any}

% Include author msg. just to stop latex from complaining
%\authorinfo{Authors removed for double-blind review}
%           {}
%           {}
%% \authorinfo{Name2\and Name3}
%%            {Affiliation2/3}
%%            {Email2/3}

\maketitle

\input{intro}

\chapter{\LANG language}
\label{chap:lang}

\section{Overview}
\label{sec:lang:overview}

\LANG is designed as a 2-layer language. The lower level is called an \emph{expression} language. It is an imperative language, similar to C and Matlab, for manipulating basic data types (bits, bytes, arrays, etc), described in \ref{sec:lang:expression} The higher level is
called a \emph{computational} language. It is a monadic\footnote{A monad is a structure that represents computations defined as sequences of steps} language for specifying and staging stream processor. It is described in \ref{sec:lang:computation}.

We start by giving a simple example of the scrambler in WiFi 802.11 a/g transmitter. The goal of the scrambler is to xor an input sequence with a predefined pseudo-random sequence (to make the transmitter input look random). \LANG code for the scrambler is given below:
\begin{lstlisting}[caption=Scrambler function of WiFi 802.11a/g transmitter in \LANG,
xleftmargin=15pt,
label=lst:wifi:tx:scrambler]
let scrambler() =
  @\label{scrambler:letref}@var scrmbl_st: arr[7] bit := {1,1,1,1,1,1,1}; 
  @\label{scrambler:letref2}@var tmp: bit; 
  @\label{scrambler:letref3}@var y: bit; 
@\label{scrambler:repeat}@  in repeat (
@\label{scrambler:take}@    x <- take;
@\label{scrambler:ret_start}@    execute (
      tmp := (scrmbl_st[3] ^ scrmbl_st[0]);
@\label{scrambler:shift}@      scrmbl_st[0:5] := scrmbl_st[1:6];
@\label{scrambler:feed}@      scrmbl_st[6] := tmp;
@\label{scrambler:ret_end}@      y := x ^ tmp); 
@\label{scrambler:emit}@    emit (y))
\end{lstlisting}
The scrambler is written as a sequence of operations in the
computational language. The scrambler's body declares three local
variables in lines~\ref{scrambler:letref} through
\ref{scrambler:letref2}: \lstinline|scrmbl_st|, an array of $7$ bits
that gives the current state of the shift register, and two one-bit
references: \lstinline|tmp| and \lstinline|y|.  The scrambler
\lstinline|take|s a value from the input stream
(line~\ref{scrambler:take}), then performs an imperative computation
that assigns \lstinline|tmp| the XOR of taps $3$ and $0$ in the shift
register, shifts the register state left by one, feeds \lstinline|tmp|
into position $6$ of the register, and finally returns the XOR of
\lstinline|tmp| and the input bit \lstinline|x|.

The sequence of operations inside \lstinline|execute()| primitive is written in the \LANG expression language. \LANG\ allows for arbitrary composition of the stream 
computer primitives and imperative code to create complex programs.
In the subsequent section we will describe in more detail the expression and the computation languages.



\section{Expression language}
\label{sec:lang:expression}

The low level imperative language for \LANG is used to express the basic imperative
calculation. This language is a mix of Matlab and C designed to make
the right trade-off between programmability and efficiency. It is a
strongly typed language, which allows us to simplify memory
management. It also supports array operations, like Matlab, which
allows for efficient mapping to SSE vector instructions.  






\section{Computational language}
\label{sec:lang:computation}

asdada




\chapter{Reference}


\section{Data Types}

\LANG supports the following simple data types:
\begin{description}
\item[bit] - binary type.
\item[bool] - logic type.
\item[double] - floating-point numbers.
\item[int] - 32 bits by default.
\item[int16]
\item[int32]
\item[complex] - both real and imaginary are 32 bits by default.
\item[complex16]
\item[complex32]
\end{description}

Also, arrays and structs.



\section{Expression Language}

\subsection{Unary operators}

\LANG expression language supports the following unary operators:
\begin{description}
\item[Negative number] (\lstinline|-a|): Negation of an integer expression.
\item[Logical not] (\lstinline|not a|): Negation of a boolean expression.
\item[Binary not] (\lstinline|~ a|): Negation of a binary expression.
\item[Array length] (\lstinline| length(a)|): Gives the length of an array.
\item[Cast] (\lstinline| type(a) |): Casts variable \lstinline|a| into type \lstinline|type|.
\end{description}




\subsection{Binary operators}

\LANG expression language supports the following unary operators:
\begin{description}
\item[Addition] (\lstinline|a+b|).
\item[Subtraction] (\lstinline|a-b|).
\item[Multiplication] (\lstinline|a*b|).
\item[Division] (\lstinline|a..b|).
\item[Exponentiation] (\lstinline|a**b| $= a^b$).
\item[Reminder] (\lstinline|a \% b|) is the reminder when dividing a by b.
\item[Binary shift left] (\lstinline|a << n|).
\item[Binary shift right] (\lstinline|a >> n|).
\item[Binary XOR] (\lstinline|a ^ b|).
\item[Binary AND] (\lstinline|a & n|).
\item[Binary OR] (\lstinline!a | n!).
\item[Logical AND] (\lstinline|a && n|).
\item[Logical OR] (\lstinline!a || n!).
\item[Equality relation] (\lstinline|a == b|).
\item[Inequality relation] (\lstinline|a != b|).
\item[Order relation] (\lstinline|a >= b, a ,< b, a < b, a > b|).
\end{description}




\subsection{Arrays}




\subsection{Control flow}




\subsection{I/O functions}

\LANG expression language supports the following I/O functions:
\begin{description}
\item[print]
\item[error]
\end{description}



%\balance
%\bibliographystyle{abbrvnat}
%\bibliography{wpl,mainland}


\end{document}

