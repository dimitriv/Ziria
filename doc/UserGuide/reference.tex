\chapter{Reference}


\section{Data Types}

\LANG supports the following simple data types:
\begin{description}
\item[bit] - binary type.
\item[bool] - logic type.
\item[double] - floating-point numbers.
\item[int] - 32 bits by default.
\item[int16]
\item[int32]
\item[complex] - both real and imaginary are 32 bits by default.
\item[complex16]
\item[complex32]
\end{description}

Also, arrays and structs.



\section{Expression Language}

\subsection{Unary operators}

\LANG expression language supports the following unary operators:
\begin{description}
\item[Negative number] (\lstinline|-a|): Negation of an integer expression.
\item[Logical not] (\lstinline|not a|): Negation of a boolean expression.
\item[Binary not] (\lstinline|~ a|): Negation of a binary expression.
\item[Array length] (\lstinline| length(a)|): Gives the length of an array.
\item[Cast] (\lstinline| type(a) |): Casts variable \lstinline|a| into type \lstinline|type|.
\end{description}




\subsection{Binary operators}

\LANG expression language supports the following unary operators:
\begin{description}
\item[Addition] (\lstinline|a+b|).
\item[Subtraction] (\lstinline|a-b|).
\item[Multiplication] (\lstinline|a*b|).
\item[Division] (\lstinline|a..b|).
\item[Exponentiation] (\lstinline|a**b| $= a^b$).
\item[Reminder] (\lstinline|a \% b|) is the reminder when dividing a by b.
\item[Binary shift left] (\lstinline|a << n|).
\item[Binary shift right] (\lstinline|a >> n|).
\item[Binary XOR] (\lstinline|a ^ b|).
\item[Binary AND] (\lstinline|a & n|).
\item[Binary OR] (\lstinline!a | n!).
\item[Logical AND] (\lstinline|a && n|).
\item[Logical OR] (\lstinline!a || n!).
\item[Equality relation] (\lstinline|a == b|).
\item[Inequality relation] (\lstinline|a != b|).
\item[Order relation] (\lstinline|a >= b, a ,< b, a < b, a > b|).
\end{description}




\subsection{Arrays}




\subsection{Control flow}




\subsection{I/O functions}

\LANG expression language supports the following I/O functions:
\begin{description}
\item[print]
\item[error]
\end{description}
